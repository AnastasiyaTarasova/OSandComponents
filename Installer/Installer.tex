\documentclass[12pt,a4paper]{article}
\usepackage[utf8]{inputenc}
\usepackage[russian]{babel}
\usepackage{amsmath}
\usepackage{amsfonts}
\usepackage{amssymb}
\usepackage{graphics}
\usepackage[pdftex]{graphicx}
\usepackage{lscape}
\usepackage{listings}
\usepackage{geometry} % Меняем поля страницы
\geometry{left=2cm}% левое поле
\geometry{right=1.5cm}% правое поле
\geometry{top=2cm}% верхнее поле
\geometry{bottom=15mm}% нижнее поле
\renewcommand{\baselinestretch}{1.5}
\pagestyle{plain}
\begin{document}
\thispagestyle{empty}
\begin{center}
\large Санкт-Петербургский государственный политехнический университет\\
Институт информационных технологий и управления\\
Кафедра компьютерных систем и программных технологий\\
\vspace{65mm}
\Large Отчёт по лабораторной работе №1\\ По предмету "Проектирование ОС и компонентов" \\на тему:\\
\LARGE\textbf{Загрузчик}
\end{center}

\vspace{40mm}
\begin{flushright}
\large Выполнила: студентка группы 53501/3\\ Тарасова А. А.\\ Преподаватель: Душутина Е. В.
\end{flushright}
\vspace{30mm}

\begin{center}
Санкт-Петербург\\ 2015
\end{center} % это титульный лист
\newpage
\tableofcontents
\newpage
Проанализировать инсталляцию ОС Linux,сформировать инсталлятор с минимальным набором функций. В системный редактор всего этого написать command file необходимого сброса. Можно  из оболочки user mode или свой вариант, который сделать сборку. Описание есть у  Fedora.
\section{Процесс загрузки Debian}
\subsection{Программа init}
Как и все Unix-подобные операционные системы, Debian загружается, выполняя программу init. Конфигурационный файл для init (/etc/inittab) указывает, что первый скрипт, который должен быть выполнен, - это скрипт /etc/init.d/rcS. Данный скрипт запускает все скрипты из каталога /etc/rcS.d/ по порядку или порождая подчиненные процессы с целью инициализации системы, как то проверка и монтирование файловых систем, загрузка модулей, запуск сетевых сервисов, установка системных часов и пр. Далее (для совместимости) этот скрипт выполняет файлы из каталога /etc/rc.boot/ (кроме тех, что имеют `.' в имени файла). Любые скрипты из последнего каталога обычно предназначаются для использования системным администратором, и применение их в пакетах не допускается.
\subsection{Уровни выполнения}
По окончании процесса загрузки программа init выполняет все стартовые скрипты в каталоге, определяемым уровнем выполенения по умолчанию (это уровень выполнения задается элементом id в файле /etc/inittab). Как и большинство System V - совместимых Unix-систем, Линукс имеет 7 уровней выполнения:
\begin{itemize}
  \item 0 (остановка системы),
  \item 1 (однопользовательский режим),
  \item 2 - 5 (различные многопользовательские режимы),
  \item 6 (перезагрузка системы)
\end{itemize}
Системы Debian идут с id=2, который показывает, что при входе в многопользовательский режим уровень выполнения по умолчанию - второй, и поэтому требуется выполнить скрипты из каталога /etc/rc2.d/.

Скрипты в любом из каталогов /etc/rcN.d/, по сути, являются символическими ссылками на скрипты из каталога /etc/init.d/. Однако, сами имена файлов в каждом /etc/rcN.d/ каталоге определяют способ, которым будут запущены скрипты из /etc/init.d/. Конкретнее, перед входом в любой уровень исполнения сначала запускаются все скрипты, начинающиеся с буквы `K'; данные скрипты останавливают сервисы. Далее выполняются все скрипты, начинающиеся в буквы `S'; эти скрипты запускают сервисы. Двузначное число после буквы `K' или `S' показывает порядок, в котором выполняются скрипты. Скрипты с меньшим номером выполняются первыми.

Это способ работает, так как все скрипты из каталога /etc/init.d/ принимают аргумент с одним из возможных значений "start", "stop", "reload", "restart" или "force-reload", и выполняют задачу, соответствующую значению данного аргумента. Эти скрипты могут также использоваться после загрузки системы для управления различными процессами.

Например, следующая команда с аргументом "reload"

\verb+     # /etc/init.d/sendmail reload+

посылает сервису сигнал, побуждающий его перечитать конфигурационный файл.
\subsection{Настройка процесса загрузки}
Debian не использует свойственный BSD каталог rc.local для настройки процесса загрузки; вместо этого он предоставляет следующий механизм.

Предположим, системе нужно на этапе загрузки или при входе в определенный (System V) уровень выполнения исполнить скрипт foo. Тогда  необходимо сделать следующее:
\begin{enumerate}
  \item Разместить скрипт foo в каталоге /etc/init.d/.
  \item Выполнить Debian-команду update-rc.d с соответствующими аргументами, чтобы создать символические ссылки между каталогами rc?.d (задаваемый в командой строке) и файлом /etc/init.d/foo. Здесь ? - это номер от 0 до 6, который соответствует одному из System V уровней выполнения.
  \item Перезагрузить систему.
\end{enumerate}
Команда update-rc.d установит ссылки между файлами в каталоге rc?.d и скриптом из /etc/init.d/. Каждая ссылка будет начинаться с `S' или `K' с последующим номером и именем скрипта. Когда система входит в уровень выполнения N, из каталога /etc/rcN.d/ скрипты, начинающиеся с `K', запускаются с аргументом stop, а потом оттуда же скрипты, начинающиеся с `S', запускаются с аргументом start.

Например, можно настроить, чтобы скрипт foo выполнялся при загрузке, разместив его в каталог /etc/init.d/ и установив ссылки при помощи команды update-rc.d foo defaults 19. Аргумент defaults ссылается на уровни выполнения по умолчанию, которые могут быть от 2 до 5. Аргумент 19 обеспечивает, что скрипт foo вызывается до любых других с номером 20 или больше.
\section{Разметить диск}
\section{Поднять сеть}
\section{Поставить пакеты}
\section{Первичные настройки}

\end{document} 