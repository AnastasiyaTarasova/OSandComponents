\documentclass[12pt,a4paper]{article}
\usepackage[utf8]{inputenc}
\usepackage[russian]{babel}
\usepackage{amsmath}
\usepackage{amsfonts}
\usepackage{amssymb}
\usepackage{graphics}
\usepackage[pdftex]{graphicx}
\usepackage{lscape}
\usepackage{listings}
\usepackage{geometry} % Меняем поля страницы 
\geometry{left=2cm}% левое поле 
\geometry{right=1.5cm}% правое поле 
\geometry{top=2cm}% верхнее поле 
\geometry{bottom=15mm}% нижнее поле
\RequirePackage{float}
\renewcommand{\baselinestretch}{1.5}
\pagestyle{plain}
\begin{document}
\include{titleWorkingWithTools} % это титульный лист
\newpage
\tableofcontents
\newpage
\section{VirtualBox}
	\textbf{VirtualBox} – это мощное кросс-платформенное средство для программной виртуализации на платформах на базе х86. «Кросс-платформенность» означает, что VirtualBox может быть установлен на компьютеры с MS Windows, Linux, Mac OS X и Solaris x86, а «средство для программной виртуализации» означает, что Вы можете создавать и запускать различные виртуальные машины, использующие различные операционные системы одновременно на одном компьютере. Например, Вы можете запустить Windows и Linux на Mac, или Linux и Solaris из-под Windows, или Windows из-под Linux.

Oracle VM VirtualBox доступна для загрузки в виде открытого кода или в виде установочных бинарных файлов для Windows, Linux, Mac OS X и Solaris.

Дистирибутив VirtualBox можно скачать по ссылке https://www.virtualbox.org/wiki/Downloads
\subsection{Создание виртуальной машины}
Внегний вид VirtualBox представлен на рисунке 1.
\begin{figure}[h!]
\centering
\includegraphics[scale=0.5]{res/VirtualBox}
\caption{VirtualBox}
\end{figure}

Для того чтобы создать виртуальную машину необходимо нажать кнопку \verb+Создать+. В окне, изображенном на рисунке 2 пишем название системы и выбираем её тип. В моём случае имя машины \verb+Debian+, тип \verb+Linux+, версия \verb+Debian (64bit)+.
\begin{figure}[h!]
\centering
\includegraphics[scale=0.5]{res/CreateMaschine}
\caption{Создание виртуальной машины, выбор типа и версии}
\end{figure}

Затем, необходимо указать объем оперативной памяти (RAM), которая будет выделена данной виртуальной машине (рисунок 3).
\begin{figure}[h!]
\centering
\includegraphics[scale=0.5]{res/RAMSize}
\caption{Выбор размера оперативной памяти}
\end{figure}

После этого необходимо определиться с жестким диском для виртуальной машины. В моем случае требуется \verb+Создать новый виртуальный жесткий диск+. Это показано на рисунке 4.
\begin{figure}[h!]
\centering
\includegraphics[scale=0.5]{res/ROM}
\caption{Выбор жесткого диска}
\end{figure}

Следующим этапом является выбор типа жесткого диска виртуальной машины.
Файлы дисковых образов располагаются на хост системе и определяются гостевыми системами как жесткие диски определенного размера. При чтении или записи данных с диска гостевой ОС, VirtualBox перенаправляет дисковые запросы к файлу образа.

Как и у физического диска, у виртуального диска есть размер, который надо указать при создании. В отличии от физических дисков, VirtualBox позволяет увеличивать образ файла после его создания.\\
VirtualBox поддерживает 4 типа файлов образов диска:
\begin{itemize}
\item Virtual Disk Image (VDI)-файлы - собственный формат виртуальных дисков. 
\item VMDK - формат, использующийся в множестве других продуктах виртуализации (VMware).
\item VirtualBox таже полностью поддерживает формат VHD разработанный Microsoft.
\item Файлы образов Parallels 2 версии (HDD format) также поддерживаются. Новые версии этого формата (3 and 4) не поддерживаются.
\end{itemize}

По умолчанию будет использоваться нативный для VirtualBox формат — \verb+VDI+.

\begin{figure}[h!]
\centering
\includegraphics[scale=0.45]{res/TypeOfHardDrive}
\caption{Выбор типа жесткого диска}
\end{figure}

Следующим шагом будет выбор между динамическим и фиксированным типом диска. 

Независимо от формата виртуальных дисков, существует два типа создаваемых образов: фиксированного размера и динамически расширяемые.
\begin{itemize}
\item Если вы создаете образ фиксированного размера емкостью 10 GB, то на хост системе будет создан файл примерно такого же размера. Создание фиксированных образов может занять довольно значительное время, в зависимости от размера образа и производительности дисковых операций системы.
\item Для более гибкого управления виртуальными носителями, используются динамически расширяемые образы . При создании данный образ будет иметь небольшой размер, за счет неиспользуемого пространства виртуального диска, но по мере использования, файл образа будет увеличиваться. Данный вид файла занимает меньше места на начальном этапе, однако VirtualBox необходимо увеличивать размер образа (пока образ не достигнет максимального размера), что ведет к замедлению дисковых операций по сравнению с дисками фиксированного размера. Однако, после достижения предела расширения динамического диска, потери производительности операций чтения и записи уже не так значительны.
\end{itemize}

В моем случае выбран \verb+динамический виртуальный жесткий диск+
\begin{figure}[h!]
\centering
\includegraphics[scale=0.45]{res/StorageFormat}
\caption{Выбор формата хранения}
\end{figure}

После нажатия кнопки \verb+Создать+  будет создан виртуальный диск с выбранными параметрами.

\begin{figure}[h!]
\centering
\includegraphics[scale=0.45]{res/WhereToInstall}
\caption{Выбор имя и размер файла виртуальной машины}
\end{figure}
\newpage
\subsection{Установка операционной системы на виртуальный диск}
Для начала следует зайти в \verb+Свойства+ и выбрать раздел \verb+Настроить+ (рисунок 8).
\begin{figure}[h!]
\centering
\includegraphics[scale=0.45]{res/Properties}
\caption{Заходим в настройки виртуальной машины}
\end{figure}

\begin{figure}[h!]
\centering
\includegraphics[scale=0.45]{res/carriers}
\caption{Заходим в настройки виртуальной машины}
\end{figure}

В блоке \verb+Контроллер ОС+ следует выбрать \verb+Пусто+, а в блоке \verb+Атрибуты+  нужно нажать на иконку CD диска. Появится список с требуемыми действиями. Если инсталляционный диск записан на CD/DVD диске, то надо выбрать пункт \verb+Привод хоста ‘Х:‘+, где Х это буква диска. В случае если ОС хранится в ISO-образе, то надо выбрать пункт  \verb+Выбрать образ оптического диска+ и указать на файл образа(рисунки 9, 10, 11).


\begin{figure}[h!]
\centering
\includegraphics[scale=0.45]{res/choose}
\caption{Заходим в настройки виртуальной машины}
\end{figure}

\begin{figure}[h!]
\centering
\includegraphics[scale=0.45]{res/chooseiso}
\caption{Заходим в настройки виртуальной машины}
\end{figure}

На этом настройка завершена. Никаких других настроек, в большинстве случаев, производить не надо. И сейчас уже можно запустить нашу виртуальную машину нажав на иконку «Старт». После процесса установки получим операционную систему работающую в виртуальной машине.

\newpage
\section{Дистрибутив Debian}
Дистрибутив Debian можно скачать по ссылке https://www.debian.org/distrib/netinst
\section{Bash}

\end{document}